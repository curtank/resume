% !TEX TS-program = xelatex
% !TEX encoding = UTF-8 Unicode
% !Mode:: "TeX:UTF-8"

\documentclass{resume}
\usepackage{zh_CN-Adobefonts_external} % Simplified Chinese Support using external fonts (./fonts/zh_CN-Adobe/)
% \usepackage{NotoSansSC_external}
% \usepackage{NotoSerifCJKsc_external}
% \usepackage{zh_CN-Adobefonts_internal} % Simplified Chinese Support using system fonts
\usepackage{linespacing_fix} % disable extra space before next section
\usepackage{cite}

\begin{document}
\pagenumbering{gobble} % suppress displaying page number

\name{张哲愚}

\basicInfo{
  \email{curtank2@gmail.com} \textperiodcentered\ 
  \phone{(+86) 151-2273-1896} \ 
  }
 
\section{\faGraduationCap\  教育背景}
\datedsubsection{\textbf{南开大学}, 天津}{2014 -- 2018}
\textit{学士}\ 软件工程专业


\section{\faUsers\ 项目经历}
\datedsubsection{\textbf{图森未来}  北京}{2017年10月 -- 至今}
\role{实习} {}
自研自动驾驶通信中间件开发
\begin{itemize}
  \item 参与公司自研通信中间件的开发,基于protobuf开发了自动驾驶系统的消息格式
  \item 在此基础上,开发了消息的存储和读取模块,支持不同版本的消息兼容,提交了多个性能改进
\end{itemize}
\role{}{}
自动驾驶系统框架开发
\begin{itemize}
  \item 基于公司内部的自研通信中间件,开发了一整套自动驾驶系统框架,定义了节点的生命周期,节点的状态管理,节点的消息处理等模块,提供了一整套完整的节点开发和管理的解决方案
  \item 该框架通过合适的抽象,使得节点的开发者可以专注于节点的业务逻辑,不需要处理节点的生命周期和消息处理,可以使用纯函数式编程的方式,提高了节点的可测试性
  \item 该框架在RPC模式下使用capnp作为消息格式,实现了消息的零拷贝,提高了消息的传输效率

\end{itemize}

\role{实习}{经理: 高富帅}
基于QNX系统的自动驾驶系统框架开发
\begin{itemize}
  \item 在Nvidia Orin SoC上开发了一整套自动驾驶系统框架,包括了通信中间件,节点框架,节点管理等
  \item 节点框架实现跨平台编译,一套代码可以在linux和QNX上编译运行
  \item 节点框架和通信中间件通过了AutoSar c++11标准
\end{itemize}

自动驾驶系统集成测试平台
\begin{itemize}
  \item 基于新的自动驾驶系统框架,开发了一套可复现的确定性的自动驾驶系统集成测试工具,
  \item 后台资源占用率减少8\%
  \item xxx
\end{itemize}
\datedsubsection{\textbf{分布式科学上网姿势}}{2014年6月 -- 至今}
\role{Golang, Linux}{个人项目,和富帅糕合作开发}
\begin{onehalfspacing}
分布式负载均衡科学上网姿势, https://github.com/cyfdecyf/cow
\begin{itemize}
  \item 修复了连接未正常关闭导致文件描述符耗尽的 bug
  \item 使用Chord 哈希 URL, 实现稳定可靠地分流
  \item xxx (尽量使用量化的客观结果)
\end{itemize}
\end{onehalfspacing}

\datedsubsection{\textbf{\LaTeX\ 简历模板}}{2015 年5月 -- 至今}
\role{\LaTeX, Python}{个人项目}
\begin{onehalfspacing}
优雅的 \LaTeX\ 简历模板, https://github.com/billryan/resume
\begin{itemize}
  \item 容易定制和扩展
  \item 完善的 Unicode 字体支持,使用 \XeLaTeX\ 编译
  \item 支持 FontAwesome 4.5.0
\end{itemize}
\end{onehalfspacing}

% Reference Test
%\datedsubsection{\textbf{Paper Title\cite{zaharia2012resilient}}}{May. 2015}
%An xxx optimized for xxx\cite{verma2015large}
%\begin{itemize}
%  \item main contribution
%\end{itemize}

\section{\faCogs\ IT 技能}
% increase linespacing [parsep=0.5ex]
\begin{itemize}[parsep=0.5ex]
  \item 编程语言: C == Python > C++ > Java
  \item 平台: Linux
  \item 开发: xxx
\end{itemize}

\section{\faHeartO\ 获奖情况}
\datedline{\textit{第一名}, xxx 比赛}{2013 年6 月}
\datedline{其他奖项}{2015}

\section{\faInfo\ 其他}
% increase linespacing [parsep=0.5ex]
\begin{itemize}[parsep=0.5ex]
  \item 技术博客: http://blog.yours.me
  \item GitHub: https://github.com/username
  \item 语言: 英语 - 熟练(TOEFL xxx)
\end{itemize}

%% Reference
%\newpage
%\bibliographystyle{IEEETran}
%\bibliography{mycite}
\end{document}